\documentclass[11pt,a4paper]{article}

\usepackage{graphicx}
\graphicspath{ { ./img } }
\usepackage[margin=1in]{geometry}
\usepackage{url}
\usepackage{xcolor}

\title{\textbf{CS142 Coursework II}}
\author{Artemy Bulavin \\ 1812276}
\date{}

\begin{document}

\maketitle
% A summary of your research
% A description of the algorithm/data set/system you are visualising
% A description of the process/techniques you have used to visualise it
% A justification of the visual choices you have made
% An image/set of images showing the output – you might also want to 
% include images of early work in progress
% A reflection on how effective the visualisation is

% MARKING CRITERIA:
% Well written and shows evidence of sufficient research into the topic
% Demonstrates that you have considered what makes a good visualisation
% Shows a critical reflection on the final output
% Well structured and well formatted
% Includes comprehensive and relevant references to your research and
% to your sources of inspiration.

\section*{Introduction}
\hrule
\vspace{11pt}
The aim of this visualisation is to demonstrate the process of Dijkstra's Shortest Path algorithm using 
a simple graph with weighted edges, where colours of vertices and edges change based on the different stages
of the algorithm.

\section*{Dijkstra's Shortest Path Algorithm}

Dijkstra's Shortest Path algorithm is a method for finding the shortest distance from a given
source node to all other nodes in a simple, weighted graph \cite{djikstra1959note}. The version of the algorithm
used in this visualisation is the original version, as opposed to min-priority queue implementation
\cite{fredman1987fibonacci}.

\bibliographystyle{plain}
\bibliography{ref}


\end{document}